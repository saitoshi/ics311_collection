% DOCUMENT FORMATING
\documentclass[12pt]{article}
\usepackage[margin=1in]{geometry}

% PACKAGES
\usepackage{amsmath} % For extended formatting
\usepackage{amssymb} % For math symbols
\usepackage{amsthm} % For proof environment
\usepackage{array} % For tables
\usepackage{enumerate} % For lists
\usepackage{extramarks} % For headers and footers
\usepackage{blindtext}
\usepackage{fancyhdr} % For custom headers
\usepackage{graphicx} % For inserting images
\usepackage{multicol} % For multiple columns
\usepackage{verbatim} % For displaying code
\usepackage{tkz-euclide}
\usepackage{pgfplots}
\newtheorem{theorem}{Theorem}[section]
\newtheorem*{theorem*}{Theorem}
\newtheorem{corollary}{Corollary}[theorem]
\newtheorem{lemma}[theorem]{Lemma}

% SET UP HEADER AND FOOTER
\pagestyle{fancy}
\lhead{\MyCourse} % Top left header
\chead{\MyTopicTitle} % Top center header
\rhead{\MyAssignment} % Top right header
\lfoot{\MyCampus} % Bottom left footer
\cfoot{} % Bottom center footer
\rfoot{\MySemester} % Bottom right footer
\renewcommand\headrulewidth{0.4pt} % Size of the header rule
\renewcommand\footrulewidth{0.4pt} % Size of the footer rule
% ----------
% TITLES AND NAMES 
% ----------

\newcommand{\MyCourse}{ICS 311}
\newcommand{\MyTopicTitle}{TOPIC 2,3,4}
\newcommand{\MyAssignment}{Algorithms}
\newcommand{\MySemester}{Fall 2020}
\newcommand{\MyCampus}{University of Hawaii at Manoa}
\begin{document}
\subsection*{Problem 1. Proofs of Asymptotic Bound}
\begin{itemize}
    \item[(a)]
    \begin{center}
    \begin{tabular} {|c|c|c|c|c|c|c|}
    f(n) & g(n) & O & o & $\Omega$ & $\omega$ & $\theta$ \\ 
    $4n^2$ & $4^{\lg n}$ & Y & N & Y & N & Y 
    \end{tabular}
\end{center}
\textbf{Proof}
Given that f(n) = $4n^2$ and g(n) = $4^{\lg n}$ from log properties we know that $\lg n = \log_{2} n$ and from that we can rewrite from the logarithmic properties such that $4^{\log_2 n} = n^{\log_2 4}$. Furthermore, since $\log_2 4 = 2$ then it can be rewritten $n^{\log_2 4} = n^2$ = g(n). \\
From definition of $\theta$, let $c_1, c_2, n_0 \in \mathbb{Z^{+}}$ such that $\forall n  \geq n_0$ then $\theta(g(n)) = \{f(n) : 0 \leq c_1 g(n) \leq f(n) \leq c_2 g(n)\}$. Suppose $c_1 = c_2 = 4$ then the statement therefore, so this is true for $\theta$ growth. Furthermore, from Theorem 3.1 we know that if $f(n) = \theta(g(n))$ iff $f(n) = O(g(n))$ and $f(n) = \Omega(g(n))$ and since this is an iff theorem then it goes both ways. Therefore, O and $\Omega$ holds true. \\
However, this fails o and $\omega$ growth because it does not meet the definitions. Suppose for the sake of contradiction that little o and little omega is true. \\
By taking the limit to the infinity of $\frac{f(n)}{g(n)}$ 
\begin{equation*}
    \lim_{n \to \infty} \frac{f(n)}{g(n)} = \lim_{n \to \infty} \frac{4n^2}{n^2} = 4 \neq 0 \neq \infty
\end{equation*}
Therefore, fails o and $\omega$. 
\item[(b)] 
 \begin{center}
    \begin{tabular} {|c|c|c|c|c|c|c|}
    f(n) & g(n) & O & o & $\Omega$ & $\omega$ & $\theta$ \\ 
    $2^{\lg n}$ & $\lg^2 n$ & N & N & Y & Y & N
    \end{tabular}
    \end{center}
\textbf{Proof} Given that f(n) = $2^{\lg n}$ and g(n) = $\lg^2 n$. From the logarithmic properties f(n) can be rewritten since $\lg n = \log_2 n $ and further $2^{\log_2 n} = n^{\log_2 2} = n^{1} = n$. \\
From definition of $\theta$, let $c_1, c_2, n_0 \in \mathbb{Z^{+}}$ such that $\forall n  \geq n_0$ then $\theta(g(n)) = \{f(n) : 0 \leq c_1 g(n) \leq f(n) \leq c_2 g(n)\}$. This fails the right side of inequality $f(n) \leq c_2 g(n)$ as in the real number system $\lg^2 n$ will not exceed n therefore it is not $\theta $ growth and not O or o growth. \\
However, since n will exceed $c_1\lg^2 n $ for any positive constant $c_1$ then this will be $\Omega$ growth and furthermore since 
\begin{equation*}
     \lim_{n \to \infty} \frac{f(n)}{g(n)} = \lim_{n \to \infty} \frac{n}{\lg^2 n} = \infty
\end{equation*}
from the definition of horizontal asymptote then this meets $\omega$ growth.  
\item[(c)] 
 \begin{center}
    \begin{tabular} {|c|c|c|c|c|c|c|}
    f(n) & g(n) & O & o & $\Omega$ & $\omega$ & $\theta$ \\ 
    $\sqrt{n}$ & $ n^{\sin(n)}$ & N & N & N & N & N
    \end{tabular}
    \end{center}
    \textbf{Proof} Given that f(n) = $\sqrt{n}$ and g(n) = $n^{\sin(n)}$. From exponential rules then $f(n) = \sqrt{n} = n^{\frac{1}{2}}$. Furthermore, since $\sin(n)$ has a range between $-1 \geq \sin(n) \geq 1$ then it follows that $n^{\sin(n)}$ will oscillate between $\frac{1}{n} \leq n^{\sin(n)} \leq n$. 
\end{itemize}
\clearpage
